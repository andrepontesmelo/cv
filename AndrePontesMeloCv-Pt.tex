\documentclass[]{friggeri-cv}

\usepackage{xcolor}
\usepackage{hyperref}
\hypersetup{
  allbordercolors=blue,
  pdfborderstyle={/S/U/W 1},
}

\begin{document}
\hyphenation{pos-ta-gem si-lá-bi-ca exe-cu-ção pa-ra-le-lo al-te-ra-ções co-lo-ri-das me-lhor}
\header{André}{Melo}{Engenheiro de Software} % Seu nome e profissão.

%----------------------------------------------------------------------------------------
%	BARRA LATERAL
%----------------------------------------------------------------------------------------

\begin{aside} 
\section{contato}
Belo Horizonte, Brasil
~
+55 (31) 9228 2842
+55 (31) 3046 1812
~
\href{mailto:andrepontes@gmail.com}{andrepontes@gmail.com}
\href{https://br.linkedin.com/in/andrepontesmelo}{linkedin://andrepontesmelo}
\href{https://github.com/andrepontesmelo}{github://andrepontesmelo}
%\href{http://facebook.com/andrepontesfb}{fb://andrepontesfb}
\section{fluência}
português
inglês 
\section{linguagens}
C\#
 {\color{red} $\varheartsuit$} Java\\Ruby\\Python\\C++, C
CSS, JavaScript
\section{frameworks}
AngularJS
Selenium
Capybara
Cucumber 
ASP.NET
\section{bibliotecas}
NHibernate
Crystal
\section{tecnologias}
 {\color{red} $\varheartsuit$} Linux
Amazon AWS
Virtualização
Automação
JSP
\section{bancos de dados}
MySQL/MariaDB
MSSQL
\section{IDEs}
 {\color{red} $\varheartsuit$} IntelliJ
Mono Develop
Visual Studio
\end{aside}

\section{apresentação}
Eu gosto de imaginar como o futuro será e o ajudo a construir enquanto imagino.


\section{experiência}

\begin{entrylist}


\entry
{2013--ATUAL}
{INDÚSTRIA MINEIRA DE JOIAS}
{Belo Horizonte, Brasil}
{ Desenvolvimento de software para empresa do ramo de joias.  

  \begin{itemize}
    \item Implementou novo aplicativo com menor tempo de resposta. 
    \item Reduziu o tempo de resposta ao refatorar o aplicativo existente.  
    \item Reduziu a carga de trabalho do setor financeiro com novo módulo.
    \item Implementou backup de máquina do tempo usando \textit{COW} do BtrFs. 
  \end{itemize}
  }


\entry
{2012--2013}
{GAP @ AVENUECODE}
{São Francisco, EUA}
{ Desenvolvimento de software para cliente \textit{fortune 500} usando metodologia ágil.  
  \begin{itemize}
    \item Desenvolvimento e pesquisa que integrou os testes funcionais ao pipeline de integração contínua usando \textit{capybara-webkit headless driver}.
  \end{itemize}
}



\entry
{2011--2012}
{BLOOMINGDALE'S @ AVENUECODE}
{Nova Iorque, EUA}
{ Desenvolvimento de software para cliente \textit{fortune 500} usando metodologia ágil.  

  \begin{itemize}
    \item Implementou interface para projeto \textit{checkout} usando YUI e JSP.
  \end{itemize}
}


\entry
{2010--2011}
{COTAK SISTEMAS}
{Belo Horizonte, Brasil}
{ Implementou uma biblioteca autonômica para configuração e log de aplicações distribuídas além de um escalonador de tarefas. 

  \begin{itemize}
    \item Liberou a máquina do cliente de executar intensa automação ao mover a execução para instâncias controladas na nuvem.
    \item Melhorou a taxa de sucesso ao implementar um painel de monitoramento.      
  \end{itemize}
}


\entry
{2009}
{CONTRATANET}
{Belo Horizonte, Brasil}
{Implementou método de \textit{matching} entre vagas e candidatos. Utilização \textit{map-reduce} para validar as alterações em testes de regressão.
}


 \entry
 {2004--2009}
 {INDÚSTRIA MINEIRA DE JOIAS}
{Belo Horizonte, Brasil}
 { Desenvolvimento de software para empresa do ramo de joias.
  \begin{itemize}
    \item Otimizou o processo mais lento da empresa levando o tempo de acerto de representante de duas semanas para dois dias.
  \end{itemize} }
  

\end{entrylist}

%----------------------------------------------------------------------------------------
%	SEÇÃO EDUCAÇÃO
%----------------------------------------------------------------------------------------
\section{educação}


\begin{entrylist}

\entry
{2013}
{{\normalfont Treinamento em} Competências Interpessoais}
{Dale Carnegie Training}

\entry
{2006}
{Publicação do art. {\normalfont Color image enhancement through 2D histogram equalization}}
{Budapeste, Hungria}
%------------------------------------------------

\entry
{2004--2010}
{{\normalfont Bacharelado em} Ciências da Computação}
{Universidade Federal de Minas Gerais}

\end{entrylist}



% \section{intercâmbios} 
%
% \begin{entrylist}
%
% \entry
% {2006--2007}
% {Operador {\normalfont de estacionamentos do cassino}}
% {Mohegan Sun Casino, Connecticut, EUA}
%
% \entry
% {2010} 
% {Operador {\normalfont de supermercado}}
% {Publix Supermarkets, Florida, EUA}
%
% \end{entrylist}
%
% \section{premiações}
%
% \begin{entrylist}
%
% \entry
% {2003} 
% {$1^{o}$ lugar {\normalfont no $3^{o}$ campeonato mineiro de robótica}}
% {Copa SUCESU, Minas Gerais, Brasil}
%
% \end{entrylist}
%
% \section{publicações}
%
% MELO, A.; MENOTTI D.; SGARBI, E.; FACON, J.; ARAUJO, A. A. Realce de imagens coloridas através de equalização de histogramas 2D. In: Workshop of Undergraduate Students - SIBGRAPI 2005, 2005, Natal, Brasil. XVIII Brazilian Symposium on Computer Graphics and Image Processing, 2005. p. 1-8.
%
% MENOTTI D.; MELO, A.; ARAUJO, A. A.; FACON, J.; SGARBI, E. Color image enhancement through 2D histogram equalization. In: International Conference on Systems, Signals and Image Processing (IWS-SIP 2006), 2006, Budapest, Hungary. Proceedings of 13th International Conference on Systems, Signals and Image Processing (IWSSIP 2006), 2006. p. 235-238.
%
% \section{interesses}
%
% \textbf{profissionais:} computação distribuída, big data, inteligência artificial, recuperação da informação, Android e Linux \textbf{pessoais:} culinária, trilhas,  xadrez, forró e piano.
%
\end{document}
