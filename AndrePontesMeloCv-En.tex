

\documentclass[]{friggeri-cv}
\usepackage{xcolor}
\usepackage{hyperref}
\hypersetup{
  allbordercolors=blue,
  pdfborderstyle={/S/U/W 1},
}

%\usepackage[document]{ragged2e}

\begin{document}

\header{André}{Melo}{Software Engineer} % Your name and current job title/field

%----------------------------------------------------------------------------------------
%	SIDEBAR SECTION
%----------------------------------------------------------------------------------------

\begin{aside} 
\section{contact}
Belo Horizonte, Brazil
+55 (31) 9228 2842
+55 (31) 3046 1812
~
\href{mailto:andrepontes@gmail.com}{andrepontes@gmail.com}
\href{https://br.linkedin.com/in/andrepontesmelo}{linkedin://andrepontesmelo}
\href{https://github.com/andrepontesmelo}{github://andrepontesmelo}
%\href{http://facebook.com/andrepontesfb}{fb://andrepontesfb}
\section{languages}
portuguese
english
\section{programming}
C\#
 {\color{red} $\varheartsuit$} Java\\Ruby\\Python\\C++, C
CSS, JavaScript
\section{frameworks}
AngularJS
Selenium
Capybara
Cucumber 
ASP.NET
\section{libraries}
NHibernate
Crystal
\section{technologies}
 {\color{red} $\varheartsuit$} Linux
Amazon AWS
Virtualization
UI Automation
JSP
\section{databases}
MySQL/MariaDB
MSSQL
\section{IDEs}
 {\color{red} $\varheartsuit$} IntelliJ
Mono Develop
Visual Studio
\end{aside}
\section{intro}

I like to imagine what the future will look like and I build it in the mean time.

\section{experience}
\begin{entrylist}

%------------------------------------------------

\entry
{2013--NOW}
{INDÚSTRIA MINEIRA DE JOIAS}
{Belo Horizonte, Brazil}
% %%%find . -name '*.cs' | grep -v esigner | grep -v emporary | awk '{gsub(" ","\\ ", $0);print $0}' | xargs wc -l 
{ Software development for jewelry company.  

  \begin{itemize}
    \item Introduced a new full stack app that runs smoothly in any hardware/OS. 
    \item Reduced overall system response time by re-factoring existing app.  
    \item Reduced the workload of finance department with new commission app.
    \item Introduced a time machine backup using BtrFs copy-on-write capability. 
  \end{itemize}
  }

%------------------------------------------------

\entry
{2012--2013}
{GAP @ AVENUECODE}
{San Francisco, USA}
{ Software development for fortune 500 client using agile methodology.  
  \begin{itemize}
    \item Research and development that integrated functional tests to continuous integration pipeline using capybara-webkit headless driver.
  \end{itemize}
}

%------------------------------------------------

\entry
{2011--2012}
{BLOOMINGDALE'S @ AVENUECODE}
{New York, USA}
{ Software development for fortune 500 client using agile methodology.

  \begin{itemize}
    \item Implemented UI for checkout project using YUI and JSP.
  \end{itemize}
}
%------------------------------------------------

\entry
{2010--2011}
{COTAK SISTEMAS}
{Belo Horizonte, Brazil}
{ Build an autonomic library for configure and log distributed applications, and a map reduce library for scheduling a set of pair<algorithm, data> dynamically. 

% Worked in development of a distributed computing cloud that automate UI web % interfaces, designed to feed automatic insurance quotes. Acquired experience in  distributed web systems, software architecture and technology research. Used C\# .NET and Amazon EC2 services. Had an opportunity to build an autonomic library for configure and log distributed applications, and a map reduce library for scheduling a set of pair<algorithm, data> dynamically.
% 

  \begin{itemize}
    \item Freed the customer machine of running intense automation by moving the execution to controlled instances in the cloud.
    \item Improved the quality score index by implementing monitoring gadget.      
  \end{itemize}
}

%------------------------------------------------
\entry
{2009}
{CONTRATANET}
{Belo Horizonte, Brazil}
{ Development of new set of classes to generate matching and ranking between jobs and candidates. Using map-reduce to validate changes in regression tests. 

}

%------------------------------------------------

 \entry
 {2004--2009}
 {INDÚSTRIA MINEIRA DE JOIAS}
 {Belo Horizonte, Brazil}
 { Software development for jewelry business company.
  \begin{itemize}
    \item Improved the worse business case that used to take 2 weeks and now takes 2 days by optimizing the company overall work-flow.
  \end{itemize} }

% %------------------------------------------------

\end{entrylist}

%----------------------------------------------------------------------------------------
%	EDUCATION SECTION
%----------------------------------------------------------------------------------------
\section{education}

\begin{entrylist}
\entry
{2015-NOW}
{Android {\normalfont Developer Nanodegree}}
{Udacity}


\entry
{2013}
{Communication \& Interpersonal {\normalfont Skills Training}}
{Dale Carnegie Training}


\entry
{2006--2006}
{Article publication {\normalfont Color image enhancement through 2D histogram equalization}}
{Budapest, Hungary}
%------------------------------------------------


\entry
{2004--2010}
{{\normalfont Bachelor of} Computer Science}
{Federal University of Minas Gerais, Brazil}
%------------------------------------------------

\end{entrylist}




% \section{work and travel} 
%
% \begin{entrylist}
%
% \entry
% {2006--2007}
% {parking lot {\normalfont operator}}
% {Mohegan Sun Casino, Connecticut, USA}
%
% \entry
% {2010} 
% {supermarket {\normalfont operator}}
% {Publix Supermarkets, Florida, USA}
%
% \end{entrylist}
%
%
% \section{awards}
%
%
% \begin{entrylist}
%
% \entry
% {2003} 
% {champion {\normalfont in robotics championship}}
% {Copa SUCESU, Minas Gerais, Brazil}
%
% \end{entrylist}
%
% \section{publications}
%
% MELO, A.; MENOTTI D.; SGARBI, E.; FACON, J.; ARAUJO, A. A. Realce de imagens coloridas através de equalização de histogramas 2D. In: Workshop of Undergraduate Students - SIBGRAPI 2005, 2005, Natal, Brasil. XVIII Brazilian Symposium on Computer Graphics and Image Processing, 2005. p. 1-8.
%
% MENOTTI D.; MELO, A.; ARAUJO, A. A.; FACON, J.; SGARBI, E. Color image enhancement through 2D histogram equalization. In: International Conference on Systems, Signals and Image Processing (IWS-SIP 2006), 2006, Budapest, Hungary. Proceedings of 13th International Conference on Systems, Signals and Image Processing (IWSSIP 2006), 2006. p. 235-238.
%
% \section{interests}
%
% \textbf{professionals:} distributed computing, big data, artificial intelligence, machine learning, information retrieval, Android and Linux \textbf{personals:} cooking, hiking,  chess, \textit{forró} (dance) and piano.
%
\end{document}
